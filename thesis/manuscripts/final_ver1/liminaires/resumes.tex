% Résumés (de 1700 caractères maximum, espaces compris) dans la
% langue principale (1re occurrence de l'environnement « abstract »)
% et, facultativement, dans la langue secondaire (2e occurrence de
% l'environnement « abstract ») :

% English

\begin{abstract}
This work lies in between the research domains of turbulence and image processing. The main objectives are to propose new methodologies to reconstruct small-scale turbulence from measurements at large-scale only. One contribution of this work is a review of existing methods. We also propose new models inspired from recent works in image processing to adapt them to the context of turbulence. We address two different problems. The first problem is to find an empirical mapping function between large and small scales for which regression models are a common approach. We also introduce the use of ``dictionary learning'' to train coupled representations of large and small scales for reconstruction. The second problem is to reconstruct small-scale information via the fusion of complementary measurements. The non-local means propagation model exploit the similarity of structures in the flow, while the Bayesian fusion model estimates the most probable fields given the measurements thanks to a maximum a posteriori estimate. All methods are validated and analyzed using numerical databases where fully resolved velocity fields are available. Performances of the proposed approaches are also characterized for various configurations. These results can be considered under the co-conception design framework where different experimental setups are designed to maximize the level of useful information after post-processing.
\end{abstract}

% French
\begin{abstract}
Ce travail est à la jonction de deux domaines de recherche que sont la turbulence et le traitement d'image. L'objectif principal est de proposer de nouvelles méthodologies pour reconstruire les petites échelles de la turbulence à partir de mesures grande échelle. Ce travail revisite des méthodes conventionnelles et propose de nouveaux modèles basés sur les travaux récents en traitement d'image pour les adapter à une problématique de turbulence. Le premier problème consiste à trouver une fonction de correspondance empirique entre les grandes et les petites échelles, une approche classique pour les modèles de type regression. Nous introduisons également une méthode appelée ``apprentissage de dictionnaire'' pour laquelle une représentation couplée des grandes et des petites échelles est déduite par apprentissage statistique. Le deuxième problème est de reconstruire les informations à petites échelles par fusion de plusieur mesures complémentaires. Le modèle de type ``propagation de la moyenne non-locale'' exploite la similarité des structures de l'écoulement alors que les modèles bayesiens de fusion proposent d'estimer le champ le plus probable en fonction d'informations données, on parle d’estimateur maximum a posteriori. 
Les performances des différentes approches sont validées et analysées sur des bases de données numériques pour lesquelles les informations sont disponibles à toutes les échelles. Ces résultats peuvent être utilisés dans une approche de type co-conception où il s’agit d'imaginer différents dispositifs expérimentaux définis conjointement avec les traitements numériques pour maximiser la qualité des informations obtenues après traitement.
\end{abstract}

% Production de la page de résumés :

\makeabstract
